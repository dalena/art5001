\subsection{Catalog Description}

Studio art class exploring the relationship between sound and image in diverse fields of practice such as film, video art, music videos, live performances, installations, and video games.

\subsection{Course Learning Objectives}

This course is an introduction and integration of sound and sonic compositions to advance multidisciplinary visual practices, design, communication, and art. Throughout the semester we will examine the language, context, and practice of sound production and how artists have contributed to and utilized this in their works. We will explore, compare, and contrast industry-standard/normative approaches with radical/experimental takes of various media in realtion to sound. Our aim is to establish a rich understanding of the complex and evolving environment in which artists and designers have been exploring the intimate realtions of sound and images. Students will explore technical, critical, and creative tools to realize audiovisual projects and to gain a deeper understanding of sound and media as a form of expression and communication.

In a series of weekly exercises, projects, presentations, discussions, workshops, and screenings students will explore and study the following:

\begin{itemize}
      \tightlist
      \item Principles and physics of sound: timing, composition, pitch, amplitude, timbre, etc
      \item Sound recording, editing, layering, and processing
      \item Everyday sounds as compositional materials
      \item Sound in multidisciplinary practices: visual art, cinema, performance, games, music, sculpture
      \item Dynamic relations between sound and image
      \item Effective use of audio in evoking and directing emotions
      \item Sonic experiences through architecture, installation, and immersive media
\end{itemize}

\subsection{Health and Safety Requirements}

All students, faculty and staff are required to comply with and stay up to date on all \href{https://safeandhealthy.osu.edu}{university safety and health guidance}, which includes wearing a face mask in any indoor space and maintaining a safe physical distance at all times. Non-compliance will be warned first and disciplinary actions will be taken for repeated offenses.

\subsection{Format \& Delivery}

This is a hands-on, process-oriented studio. It is comprised of presentations, assignments, participatory activities and exercises, individual and group discussions, and reviews. This course is \hl{hybrid or in-person}. Synchronous Zoom meetings will be used for the introduction of assignments, some demonstrations, breakout group meetings, and group critique discussions. Other activities such as working on assignments and exercises, viewing videos, and reading assignments will be executed synchronously and asynchronously. In-person activities will include demonstrations, presentations, group exercises, and critiques. Weekly announcements will serve to inform when activities will take place.

\paragraph{Departmental Note:} A hybrid course provides online learning opportunities for up to 74\% of the semester. That means that up to three-fourths of your in-class meeting time may occur at a distance with the expectation that your full attention will be given to this course during the scheduled two hour and forty minute long meeting times, regardless if you are meeting physically or otherwise.

\subsection{Attendance}

Each unexcused absence (beyong the allowed three) will result in one full letter grade deduction (e.g. B+ to C+). Six unexcused absences (20\% of the semester) results in a failed grade. If there is an emergency and you must miss class, contact us beforehand. Absences will not be excused after the fact except in extreme circumstances. Illness requires a doctor’s note. If you are more than 10 minutes late, you will be marked tardy. Three tardies result in one unexcused absence. Any disputes should be discussed within two weeks.

\paragraph{Departmental Note:} The Department of Art acknowledges that illness, family obligations, and other conflicts with your classes do occur from time to time and up to three absences are allowed for any reason during the semester without penalty. \ul{All absences from class will be counted, however, and in the instance that you miss three class meetings, you are required to meet us to discuss strategies for avoiding additional absences}.

\paragraph{Departmental Note:} It has been determined that some in-person learning is necessary for you to successfully engage your instructor and peers, course activities, and to meet learning objectives. Timely and consistent contributions are critical in all formats used to deliver the content of this course. In the instance of class-wide quarantine or campus closure, a course contingency plan has been designed so that we can transition to an exclusively on-line format if we are required to actuate one. \ul{Attendance will be taken regardless of delivery format.}


\subsection{Participation}

Attendance, productive class activity and, meeting in-progress deadlines are factors in the assessment of your progress. You are expected to be present and active for the entire class period. Participation is critical to passing and enjoying this class. Do the work, share your thoughts, ask questions, prepare for class meetings and discussions, offer feedback during critiques. This class is meant to be a safe space in which you feel encouraged and supported in learning and taking creative risks. This means being aware and considerate of different backgrounds, perspectives, and identities. Respect each other and this space we are building together. Don’t assume, ask. Remain open, be willing to take responsibility, apologize, and learn. Help each other in this. If you have concerns, please let us know.

% \subsection{Communication}\label{ssec:communication}
\hypertarget{communication}{%
      \subsection{Communication}}

\href{http://discordapp.com/}{Discord} is used as our primary mode of communication. You are required to signup for an account, join our \href{\discordURL}{server}, and keep up to date with announcements and group discussions. Discord is also used to organize resources, readings, screenings, and learning materials. Here, you will also submit your assignments.

\subsection{Discord Server Interaction}
Ongoing weekly discussions and participation in the Discord \href{\discordURL}{server} is required. We will use Discord to gather and share resources, respond to readings and peers' works, and share your work in progress.

Throughout the semester, students will submit a variety of posts to our Discord server, among them:

\begin{itemize}
      \tightlist
      \item Responses to readings in the form of text, audio, images, sketches, or video.
      \item Exercise submissions as project files, audio files, video links (uploaded to Youtube or Vimeo), rendered stills, etc.
      \item Project proposals in PDF format or as Word documents.
\end{itemize}

\subsection{Readings, Discussions \& Presentations}

During the semester, you will be assigned readings on a variety of topics. The readings are intended to familiarize you with some of the relevant discussions that relate to the field. We will discuss our findings and thoughts with our peers in class. Your participation in these discussions matters. The discussions serve as a dialectical engagement to learn from one another and explore the readings in conversation. Moreover, the readings serve as a foundation for discussing the screenings, which are purposefully picked to convey some of the ideas from the readings in practice.

\subsubsection{Required Books}

\begin{itemize}
      \tightlist
      \item \emph{\href{https://books.google.com/books?id=j-iqBgAAQBAJ&newbks=1&newbks_redir=0&source=gbs_navlinks_s}{Music, Sound and Multimedia: From the Live to the Virtual (2007)}}, Edited by Jamie Sexton | \href{supplements/Music_Sound_and_Multimedia_-_From_the_Live_to_the_Virtual_(Music_and_the_Moving_Image)_(2008).pdf}{Download PDF}
      \item \emph{\href{https://books.google.com/books/about/The_Oxford_Handbook_of_Sound_and_Image_i.html?id=XWHSAQAAQBAJ}{The Oxford Handbook of Sound and Image in Digital Media (2013)}}, Edited by Carol Vernallis, Amy Herzog, John Richardson | \href{supplements/The_Oxford_Handbook_of_Sound_and_Image_in_Digital_Media_(2013).pdf}{Download PDF}
\end{itemize}

\subsubsection{Supplemental Books}

\begin{itemize}
      \tightlist
      \item \emph{\href{https://books.google.com/books/about/Sound_and_Image.html?id=-wpUzQEACAAJ}{Sound and Image: Aesthetics and Practices (2020)}}, Edited by Andrew Knight-Hill | \href{supplements/Sound_and_Image_-_Aesthetics_and_Practices_(Sound_Design)_(2020).pdf}{Download PDF}
      \item \emph{\href{https://books.google.com/books?id=aGYPEAAAQBAJ&newbks=1&newbks_redir=0&source=gbs_navlinks_s}{Making Media: Foundations of Sound and Image Production (2017)}} by Jan Roberts-Breslin | \href{supplements/Making_Media_-_Foundations_of_Sound_and_Image_Production_(2003).pdf}{Download PDF}
      \item \emph{\href{https://books.google.com/books?id=KbYyDQAAQBAJ&dq=On_Sonic_Art_(Contemporary_Music_Studies)_(2002)&source=gbs_navlinks_s}{On Sonic Art (1996)}} by Trevor Wishart | \href{supplements/On_Sonic_Art_(Contemporary_Music_Studies)_(2002).epub}{Download EPUB}
\end{itemize}

\subsubsection{Presentations \& Discussions}

We will have four group presentations during the semester. As a group, each person, along with 1-2 others, will be responsible for leading one 30 minute discussion on that week's assigned topic. You can sign up for your preferred discussion topics by posting to \discord{Discussions}{init} with your top three choices. We will assign you to your group based on your preference and the order it was received. Your group and discussion assignments can be viewed in the \href{https://docs.google.com/spreadsheets/d/1Fw-DHUMtnCnlHasDcXCUGjx8n23QxS_RaPrshLq49R0/edit?usp=sharing}{discussion spreadsheet}. We will take the liberty of assigning you to a discussion group if you fail to sign up in time.

On these weeks, \ul{those who are not presenting are required to read/watch/listen to at least one of the posted media in advance}. The presentation topics correspond and are assigned based on the sections from \emph{\href{supplements/Music_Sound_and_Multimedia_-_From_the_Live_to_the_Virtual_(Music_and_the_Moving_Image)_(2008).pdf}{Music, Sound and Multimedia: From the Live to the Virtual}}

\textbf{For the discussion session that you lead:}
\begin{itemize}
      \tightlist
      \item Consume all posted media.
      \item Prepare a few slides to focus the discussion:
            \begin{itemize}
                  \item Include key points from the posted media
                  \item Find relevant external references and media
                  \item Examine and explore artists, practices, and projects that directly or indirectly relate to the topic
            \end{itemize}
      \item Prepare a set of questions to guide the discussion.
      \item Schedule a meeting with us in advance of your discussion to review your questions and plan. Reach out to us on Discord in advance to schedule a time to meet.
\end{itemize}

\subsection{Sample Library}

Throughout the semester, we will collect and contribute a series of field recordings to a sample library shared among your peers. This library serves as a repository using which you will create, examine, and experiment with long-format audiovisual production. During the first two weeks, you will each submit five recordings (per week) to our sample library. In addition to these early ten recordings, you will submit 14 additional recordings --- 2 per week assigned on seven weeks. By the end of the semester, this activity results in 24 recordings submitted by each of you. This should result in a library of roughly 300 sounds by week 12 of the semester.

Sounds can be recorded on your phone or using other appropriate devices. Your sounds can range anywhere in terms of length. Pay attention to interesting sounds that can operate as unique aural components: the closing of a book as percussive sounds, the ignition of a car engine as melody, or the sound of boiling eggs as rhythm; your imagination is the limit! Name your sound files as \texttt{[TYPE] Recording RAW/CLEAN (Firstname Lastname)}. For example, I might submit a file as \texttt{[DRUMS] Boiling Eggs RAW (Hirad Sab)}. The \texttt{[TYPE]} is for you to decide based on the most appropriate category. As we learn how to master and clean our recordings, we will submit our sounds with the \texttt{CLEAN} tag, e.g. \texttt{[DRUMS] Boiling Eggs \textbf{CLEAN} (Hirad Sab)}

\subsection{Projects}

Projects are due at the start of class on the date assigned. Projects may be turned in up to one week late for a one letter grade deduction off the project grade. Work that is more than one week late will not be accepted. If you are absent, you are still expected to turn in projects online by the deadline. Extra time will not be given for work lost due to save issues, software errors, computer crash, etc. You should regularly backup your files on your desktop, online, and/or on an external harddrive or USB stick in case your computer is lost.

\subsection{Grading}

There are 100 possible points, distributed across participation, attendance, exercises, and projects. There are 8 additional extra credit points available through challenges. Individual works will be assessed according to assignment objectives, effort and quality of in-class and online or distance activities, vigor of exploration and research initiative, participation in reviews and discussions, and ability to adapt.

\hspace*{1em} Participation \& Interaction: 10 pts\\
\hspace*{1em} Exercises: 20 pts\\
\hspace*{1em} Sound Library: 20 pts\\
\hspace*{1em} Project Prototype: 20 pts\\
\hspace*{1em} \ul{Project Final: 30 pts}\\
\hspace*{1em} \textbf{Total}: 100 pts\\
\hspace*{1em} \textbf{Extra credit}: 8 pts (from challenges)

\subsection{Late Assignments}

If you miss deadlines due to valid, extenuating circumstances you may submit the required work at a date agreed upon with us. Please contact us to discuss modifying the deadline prior to the original deadline.

\subsection{Grading Scale}

\begin{tabularx}{\textwidth}{@{}l @{}l X@{}}
      A \hspace*{1em} & (93 - 100) & Work, initiative, and participation of exceptional quality             \\
      A-              & (90 - 92)  & Work, initiative and participation of very high quality                \\
      B+              & (87 - 89)  & Work, initiative and participation of high quality                     \\
      B               & (83 - 86)  & Very good work, initiative and participation                           \\
      B-              & (80 - 82)  & Slightly above average work, initiative and participation              \\
      C+              & (77 - 79)  & Average work, initiative and participation                             \\
      C               & (73 - 76)  & Adequate work; less than average level of initiative and participation \\
      C-              & (70 - 72)  & Passing but below good academic standing; less than average level      \\
      D+              & (67 - 69)  & Below average work, initiative and participation                       \\
      D               & (60 - 66)  & Well below average work, initiative and participation                  \\
      E               & (59.9 - 0) & Unsuccessful completion of work. Limited or no participation.
\end{tabularx}


\subsection{Course Technology}

\begin{itemize}
      \tightlist
      \item Basic computer and \href{https://lmgtfy.com/}{web-browsing skills}
      \item Navigating Carmen: for questions about specific functionality, see the \href{https://community.canvaslms.com/docs/DOC-10701}{Canvas Student Guide}.
      \item \href{https://go.osu.edu/Bqdx}{CarmenZoom Virtrual Meetings}
      \item \href{http://discordapp.com/}{Discord} usage and interaction skills
\end{itemize}

\subsection{Required Equipment}

\begin{itemize}
      \tightlist
      \item Computer: OS X, Windows 7+, or Linux with internet connection for CarmenZoom
      \item Minimum Hardware Requirements:
            \begin{itemize}
                  \item 64-bit quad core CPU
                  \item 16 GB RAM
                  \item Full HD display
                  \item Graphics card with 4 GB RAM
            \end{itemize}
      \item \href{https://amzn.to/2XyXzc9}{\textbf{Studio headphones}}
            \begin{itemize}
                  \item \textbf{Note:} Although not required it is highly recommended that you use high-fidelity studio monitors or headphones. You can purchase a nice studio headphone for a reasonable price from the link above. Find more information and read about industry-recommended budget headphones \href{https://www.gearank.com/guides/cheap-studio-headphones}{here} and \href{https://www.musicradar.com/news/best-budget-studio-headphones}{here}. Audio-Technica, Sennheiser, AKG, and Samson are among trusted brands producing quality budget headphones under the \$50 and \$100 price range.
            \end{itemize}
      \item Webcam
      \item Microphone
      \item A mobile device (smartphone or tablet) or landline to use for BuckeyePass authentication
\end{itemize}

\subsection{Course Materials and Tools}

Our course heavily relies on free, open-source, and libre software. Throughout the semester, we will explore the aural and visual realms through a variety of software and techniques. We will primarily use \href{https://www.reaper.fm/}{Reaper} as our digital audio workstation (DAW) as we explore different audio-plugins, virtual modular synthesizers such as \href{https://vcvrack.com/Rack}{VCV Rack}, audio analysis tools such as \href{https://www.adobe.com/products/audition.html}{Adobe Audition} and \href{https://www.audacityteam.org/}{Audacity}, while also discussing other established and emerging tools aimed at creative and experimental sound design (\href{https://cycling74.com/products/max}{Max}, \href{https://puredata.info/}{Pure Data}, \href{https://derivative.ca/}{TouchDesigner}, \href{http://www.borderlands-granular.com/app/}{Borderlands Granular}, \href{https://sonic-pi.net/}{Sonic Pi}, etc.) As this is a high-level special topic course, and given the department's heavily visual-oriented curriculum, we assume that you are already well-versed in visual thinking and the production of moving images. Thus students are allowed to freely choose their software, techniques, and medium of visual production. Nevertheless, in our demonstrations, we will explore how sound can build upon, augment, complement, and complicate visual artifacts. To achieve this, we will use different digital content creation (DCC) suites such as \href{http://blender.org/}{Blender}, \href{https://www.blackmagicdesign.com/products/davinciresolve/}{DaVinci Resolve}, and \href{https://derivative.ca/}{TouchDesigner} to demonstrate concepts as they relate to sound and image.


\href{http://discordapp.com/}{Discord} is used as our primary mode of communication. You are required to signup for an account, join our \href{\discordURL}{server}, and keep up to date with announcements and group discussions. Discord is also used to organize resources, readings, screenings, and learning materials. Here, you will also submit your assignments.

You are required to signup for \href{https://www.youtube.com/}{YouTube} or \href{https://vimeo.com/}{Vimeo}. These platform are used to share your audiovisual work.

All required readings and screenings will be posted on our Discord \href{\discordURL}{server}. There is no required book for this class. We will coordinate and discuss with the department the possibilitites of lab computer use. However, given our current post-COVID reality, this course is structured such that projects and exercises can be completed with consumer-grade PCs and laptops.