\def\dMon{Mon, 01/25}
\def\dTues{Tues, 01/26}
\def\dWed{Wed, 01/27}
\def\dThur{Thur, 01/28}
\def\dFri{Fri, 01/29}
\def\dSat{Sat, 01/30}
\def\dSun{Sun, 01/31}
\placeDate

\begin{itemize}[noitemsep,topsep=0pt,leftmargin=*]
	\item \itemHead{Workshops:} Effects, Noise Reduction, Importing \& Exporting, Project Management, Envelopes, Mastering
	\item \itemHead{Reading \& Presentation}[The Sound of Cinema][Response Due \dMon]
	      \begin{itemize}
		      \item \href{supplements/MICHEL_CHION_PROJECTIONS_OF_SOUND_ON_IMAGE.pdf}{\emph{Projections of Sound on Image}}, Michel Chion
		      \item \textbf{Prompt:} What does Chion mean by ``added value''? What is sound's relationship to time and the image? Choose a scene from a film and describe how both the moving image and the audio work together or what it would be like without one or the other. Respond with a minimum of 100 words. Submit your response to \discordR
	      \end{itemize}
	\item \itemHead{Exercise}[Re-narrativize][Due \dSun] \newline
	      Select a footage of minimum 1:30-minute length. This can be a scene/shot from one of you favorite movies --- or perhaps one that you despise --- a music video, segment from the news, etc. For this exercise you are to remove the audio from this footage and replace it with one that you reimagine (using our class sound library or any of the resources from week 2). We are most interested in work that through reconstruction of audio, dramatically changes the original mood of the video. Upload you videos to Youtube or Vimeo and submit the link to \discordE
	\item \itemHead{Sample Library}[Cleanup][Due \dSun] \newline
	      Re-submit your field recordings from last week to the \href{\samplelibPermURL}{sample library}. Use the techniques discussed in class to reduce noise and master your sounds appropriately.
	      \begin{itemize}
		      \item \textbf{Filename:} \texttt{[TYPE] Recording CLEAN (Firstname Lastname)}
		      \item \textbf{Specs:} 24bit 44100hz WAV File
		      \item \textbf{Track Submission:} Submit your file names to \discordS
	      \end{itemize}
	\item \itemHead{Sample Library}[Additions][Due \dSun] \newline
	      Submit a minimum of 5 additional field recordings to the \href{\samplelibPermURL}{sample library}.
	      \begin{itemize}
		      \item \textbf{Filename:} \texttt{[TYPE] Recording CLEAN (Firstname Lastname)}
		      \item \textbf{Specs:} 24bit 44100hz WAV File
		      \item \textbf{Track Submission:} Submit your file names to \discordS
	      \end{itemize}
	\item \itemHead{Screenings/Artists:}
	      \begin{itemize}
		      \item \href{https://www.youtube.com/watch?v=u0pEpA_Y1a4}{\emph{Tango}} (1980), Zbigniew Rybczyński
		      \item \href{https://www.youtube.com/watch?v=esfUwg1-xrI}{Sound Fields: Adventures in contemporary field recording}
		      \item \href{https://vimeo.com/64302190}{Bill Viola: The Tone of Being}
		      \item \href{https://www.youtube.com/watch?v=XwjlYpJCBgk}{Ryoji Ikeda: Test Pattern 100m Version at Ruhrtriennale 2013}
		      \item \href{https://www.youtube.com/watch?v=-DksmbDMDUU}{\emph{Kaleidoscope, A Colour Box, Colour Flight}} (1935-1937), Len Lye
		      \item \href{https://www.youtube.com/watch?v=Y8Zpr1ESEO4}{Haroon Mirza: Artist Interview}
		      \item \href{https://www.youtube.com/watch?v=ZS4Bpr2BgnE}{\emph{Can’t Help Myself}} (2016), Sun Yuan and Peng Yu
	      \end{itemize}
	\item \itemHead{Resources:}
	      \begin{resenv}
		      \begin{itemize}
			      \item \href{https://www.youtube.com/watch?v=TKBzjSSaKXU}{Recording Sound on Location}
			      \item \href{https://www.youtube.com/playlist?list=PLM0xHqxaiT6-plorG47t3balft4nVki39}{Reaper 6 Mini Course}
			      \item \href{https://www.youtube.com/watch?v=7BfWqRapF5E}{Mastering Field Recordings in REAPER}
		      \end{itemize}
	      \end{resenv}
\end{itemize}