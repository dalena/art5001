\def\dMon{Mon, 01/25}
\def\dTues{Tues, 01/26}
\def\dWed{Wed, 01/27}
\def\dThur{Thur, 01/28}
\def\dFri{Fri, 01/29}
\def\dSat{Sat, 01/30}
\def\dSun{Sun, 01/31}
\placeDate

\begin{itemize}[noitemsep,topsep=0pt,leftmargin=*]
	\item \itemHead{Workshops:} Effects, Noise Reduction, Importing \& Exporting, Project Management, Envelopes, Mastering
	      %     \item \itemHead{Prepare}[Megascan Access][Due \dMon] Follow the guide on \discord{General}{resources} to prepare you account for \href{https://quixel.com/megascans}{Megascans} with unlimited access.
	      %     \item \itemHead{Proposal}[Semester-long Project][Due \dWed] This assignment only applies to \ul{those who are repeating this course for a second time}. If this is the first time you are taking 4401, this is NOT you! Moreover, if you are repeating this course and want to follow the standard syllabus this does not apply to you either. For repeating students who are pursuing a semester-long project, you must submit a proposal outlining your plan, concepts, ideas, and software. Your proposal must include:
	      %           \begin{itemize}
	      %               \item 500-word description of your idea and concept
	      %               \item Include the approximate length of your animation.
	      %               \item Rough timeline of deliverables and development plan for the next 13 weeks.
	      %               \item A detailed storyboard, if your project is narrative work, or a mood board if your project does not follow conventional linear storytelling and narratives.
	      %               \item A list of artists whose work you find inspiring in realizing your own project. These can be curated based on aesthetics, technique, software, and/or concept.
	      %               \item List of the software you are using and how you plan to use them.
	      %           \end{itemize}
	      %           Submit your proposal as a PDF file under \discord{Proposals}{extended-projects}. \newline
	      %           \small{\textbf{Note:} As you progress through your project, we understand that things change. Creative work is ``part accident, part intention''. This proposal enables us to better assist you in realizing your project and to follow and track your progress along the way. It is not a binding contract, so don't worry if things change.}
	      %     \item \itemHead{Exercise}[3D Forage][Due \dSun] Submit 3 renders of your environment and the creature placed within it based on your progress during the workshops and an additional 3 renders based on the changes that you have made to your model outside of class. If you are using resources outside of \href{https://quixel.com/megascans}{Megascans} credit them appropriately. Submit your 6 rendered images to \discordE
	      %     \item \itemHead{Screenings/Artists}
	      %           \begin{itemize}
	      %               \item \emph{Regular Division}, Joe Hamilton
	      %               \item \emph{BREATHE DEEP}, Katie Torn
	      %               \item \emph{insight}, Kim Laughton
	      %           \end{itemize}
	      %     \item \itemHead{Resources:} \href{https://www.youtube.com/watch?v=whPWKecazgM}{\emph{World Building in Blender}, Ian Hubert}, \href{https://rhizome.org/art/artbase/}{Rhizome Artbase}
	\item \itemHead{Reading \& Presentation}[The Sound of Cinema][Response Due \dMon]
	      \begin{itemize}
		      \item \href{supplements/MICHEL_CHION_PROJECTIONS_OF_SOUND_ON_IMAGE.pdf}{\emph{Projections of Sound on Image}}, Michel Chion
		      \item \textbf{Prompt:} What does Chion mean by ``added value''? What is sound's relationship to time and the image? Choose a scene from a film and describe how both the moving image and the audio work together or what it would be like without one or the other. Respond with a minimum of 100 words. Submit your response to \discordR
	      \end{itemize}
	\item \itemHead{Exercise}[Re-narrativize][Due \dSun] \newline
	      Select a footage of minimum 3-minute length. This can be a scene/shot from one of you favorite movies --- or perhaps one that you dispise --- a music video, segment from the news, etc. For this exercise you are to remove the audio from this footage and replace it with one that you reimagine (using our class sound library or any of the resources from week 2). We are most interested in works that through reconstruction of audio, dramatically change the original mood of the video. Upload you videos to Youtube or Vimeo and submit the link to \discordE
	\item \itemHead{Sample Library}[Cleanup][Due \dSun] \newline
	      Re-submit your field recordings from last week to the \href{\#}{sample library}. Use the techniques discussed in class to reduce noise and master your sounds appropriately.
	      \begin{itemize}
		      \item \textbf{Filename:} \texttt{[TYPE] Recording CLEAN (Firstname Lastname)}
		      \item \textbf{Specs:} 24bit 44100hz WAV File
		      \item \textbf{Track Submission:} Submit your file names to \discordS
	      \end{itemize}
	\item \itemHead{Sample Library}[Additions][Due \dSun] \newline
	      Submit a minimum of 5 additional field recordings to the \href{\#}{sample library}.
	      \begin{itemize}
		      \item \textbf{Filename:} \texttt{[TYPE] Recording CLEAN (Firstname Lastname)}
		      \item \textbf{Specs:} 24bit 44100hz WAV File
		      \item \textbf{Track Submission:} Submit your file names to \discordS
	      \end{itemize}
	\item \itemHead{Resources:} \href{https://www.youtube.com/watch?v=7BfWqRapF5E}{Mastering Field Recordings in REAPER}
\end{itemize}