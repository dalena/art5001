\def\dMon{Mon, 01/11}
\def\dTues{Tues, 01/12}
\def\dWed{Wed, 01/13}
\def\dThur{Thur, 01/14}
\def\dFri{Fri, 01/15}
\def\dSat{Sat, 01/16}
\def\dSun{Sun, 01/17}
\placeDate

\begin{itemize}[noitemsep,topsep=0pt,leftmargin=*]
      \item Introduction \& Discussion
      \item Claim Forms, Computer \& Door Access
      \item Syllabus Overview
      \item Logistics \& Communication
      \item Workshop
\end{itemize}
\vspace{1em}

\begin{itemize}[noitemsep,topsep=0pt,leftmargin=*]
      \item \itemHead{Logistics}[][Responses Due \dWed]
            \begin{itemize}
                  \item Complete \href{https://forms.gle/RnFhT6QbLLC9UYqm7}{Class Survey}
                  \item Fill Out \href{https://drive.google.com/file/d/1SeksTmFmQa6uehrmoSIL_uO8tuORo0P3/view}{Claims Form}
                        \newline Upload the completed form \href{https://osu.app.box.com/f/21e679a1efd4425b869edf8df0c4c77a}{here} as \texttt{Lastname-Firstname.pdf}
            \end{itemize}
      \item \itemHead{Presentation}[Introduction][In-class \dWed] \newline Create a 5 minutes long presentation about yourself. This should include your previous works, expertise, and how/where you want to move forward in your career and practice. If you don't feel comfortable sharing your projects, instead include up to 3 works that interest you conceptually, technically, and aesthetically. Your presentations must be made with \href{https://docs.google.com/presentation/}{Google Slides}. Make sure it is shared with public access rights and share this public link in our Discord server under \discord{Exercises}{presentations}
      \item \itemHead{Challenge}[Foundations for Sound Design][Due \dSun]
            \newline Your first challenge is to familiarize yourself with some of the fundamentals, practices, techniques, and skills that relate to audio. For this challenge, you are to engage with the \href{http://www.acousticslab.org/RECA220/}{Fundamentals of Sound} course by Prof. Vassilakis. Try to tackle as many of the topics as you can. The organized format of the website allows for both surface-level and in-depth examination of the topics. Use these as entry points to understand the physical and emotional properties of sound and how to experiment with them. After you are done engaging with the fundamentals, choose 4 of the topics or aspects of sound (pitch, amplitude, timbre, frequency, etc.) and write a 5-6 sentence summary for each concept. Submit your writings to our Discord server under \discordC
      \item \itemHead{Resources:} \href{https://www.kadenze.com/courses/experimental-foundations-for-sound-design/info}{Experimental Foundations for Sound Design}, \href{http://www.acousticslab.org/RECA220/}{Fundamentals of Sound by Prof. Vassilakis}
\end{itemize}