\def\dMon{Mon, 02/08}
\def\dTues{Tues, 02/09}
\def\dWed{Wed, 02/10}
\def\dThur{Thur, 02/11}
\def\dFri{Fri, 02/12}
\def\dSat{Sat, 02/13}
\def\dSun{Sun, 02/14}
\placeDate

\begin{itemize}[noitemsep,topsep=0pt,leftmargin=*]
%     \item \itemHead{Workshops:} Rendering, EEVEE Optimization, Video Encoding Fundamentals
%     \item \itemHead{Proposal}[Project 1][Response Due \dMon]
%           \begin{itemize}
%               \item \textbf{A 9-panel storyboard}: Your storyboard can be either hand-drawn, digital, or prototyped using 3D software. If time permits, explore a combination of these methods to push your sketches closer to your mental image. Your storyboard should plan for \textbf{a minimum of 1 minute of animation}.
%               \item \textbf{400-word written statement}: In your written response, explain your concept, creature, and the environment that you have in mind.How do you plan to execute future developments beyond what has already been accomplished in workshops and exercises. \newline
%                     Example: ``I'm creating an invertebrate creature in a futuristic technological dystopia\dots To achieve this my piece will primarily explore the creature as it navigates the space filled with electronic, gadgets, and trash\dots''
%               \item \textbf{References and mood board}: You should include a mood board (minimum 10 images/videos) in your proposal that that builds upon your existing creature and environment. Use these references to communicate your piece's mood, as well as materials and visual presentation.
%               \item \textbf{Sounds}: Submit a minimum of 5 audio pieces that convey the mood of your animation piece, creature, and environment. Include these as links in your PDF file.
%           \end{itemize}
%           Submit your proposal as a PDF file under \discord{Proposals}{project-1}. \newline
%           \small{\textbf{Note:} As you progress through your project, we understand that things change. Creative work is ``part accident, part intention''. This proposal enables us to better assist you in realizing your project and to follow and track your progress along the way. It is not a binding contract, so don't worry if things change.}
%     \item \itemHead{Readings}[][Response Due \dWed]
%           \begin{itemize}
%               \item \href{https://reallifemag.com/motion-pictures/}{\emph{Motion Pictures}}, Patrick Nathan
%               \item \textbf{Prompt:} Would you define Project 1 as an attempt in photography or rather a cinematic experience? Does it navigate space in the tradition of cinema, a freezing of time as do photographs, or does it echo a moment in time indefinitely? Why? What do you predict for the future of narrative in the age of 15-second videos, Instagram, and TikTok? Respond with a minimum of 200 words. Submit your response to \discordR
%           \end{itemize}
%     \item \itemHead{Screenings/Artists}
%           \begin{itemize}
%               \item \emph{still lost I guess, here’s a tunnel\dots}, Darío Alva
%           \end{itemize}
%     \item \itemHead{Resources:} \href{https://www.studiobinder.com/blog/how-to-make-storyboard/}{How to Make a Storyboard}
\item \itemHead{Reading \& Presentation}[Fandom and Music Videos][Response Due \dMon]
\begin{itemize}
	\item \emph{Section 1: Fandom and Music Videos} from \href{supplements/Music_Sound_and_Multimedia_-_From_the_Live_to_the_Virtual_(Music_and_the_Moving_Image)_(2008).pdf}{Music, Sound and Multimedia}
	\item \textbf{Prompt:} TBA. Submit your response to \discordR
\end{itemize}
\item \itemHead{Sample Library}[Additions][Due \dSun] \newline 
 Submit a minimum of 2 additional field recordings to the \href{\#}{sample library}.
\begin{itemize}
	\item \textbf{Filename:} \texttt{[TYPE] Recording CLEAN (Firstname Lastname)}
	\item \textbf{Specs:} 24bit 44100hz WAV File
	\item \textbf{Track Submission:} Submit your file names to \discordS
\end{itemize}
\end{itemize}