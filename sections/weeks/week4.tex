\def\dMon{Mon, 02/01}
\def\dTues{Tues, 02/02}
\def\dWed{Wed, 02/03}
\def\dThur{Thur, 02/04}
\def\dFri{Fri, 02/05}
\def\dSat{Sat, 02/06}
\def\dSun{Sun, 02/07}
\placeDate


\begin{itemize}[noitemsep,topsep=0pt,leftmargin=*]
	\item \itemHead{Workshops:} Timing \& Pace, MIDI, Virtual Instruments (VST), Sound Analysis
	\item \itemHead{Prepare}[Good Time (2017)][Due \dMon] \newline
	      Watch the Safdie Brothers' \href{https://www.netflix.com/watch/80191344?source=35}{Good Time} prior to our meeting on \dMon. We will discuss the use of sound in guiding the viewers' emotions during class.  \newline
	      \ul{Trigger Warning:} Sex \& Nudity, Violence \& Gore, Profanity, Alcohol, Drugs \& Smoking, Frightening \& Intense Scenes (\href{https://www.imdb.com/title/tt4846232/parentalguide}{Full List}). Reach out to us if you don't feel comfortable with such subjects.

	\item \itemHead{Proposal}[Semester-long Project][Due \dWed] \newline
	      For your \hyperlink{longproject}{semester-long project}, you must submit a proposal outlining your plan, format \& medium, concept, ideas, and tools. Compile your proposal (text document, images, storyboard, sounds, etc.) in a shared Google Drive or Box folder and submit the link to \discord{Projects}{proposals}. Your proposal must include:
	      \begin{itemize}
		      \item \textbf{500-word written statement:} In your written response, clearly explain and examine your concept, format, and approach towards your project. How do sounds and images collaborate, complicate, complement, and contemplate? What are your intentions, and how do you plan to direct and control them? What is your relationship with your audience? What are you trying to communicate? Is it an issue, a concept, an idea, or a feeling? How to plan to evoke, provoke, interact and entertain? What is the final piece supposed to be? How is it supposed to look and feel like? \textsuperscript{[\hyperlink{projformat}{Medium \& Format}, \hyperlink{projstyle}{Sound \& Image}]}
		      \item \textbf{Durational Mode:} Include the approximate length of your piece, or indicate the temporal mode if non-linear: looping, indefinite, generative, etc. \textsuperscript{[\hyperlink{projduration}{Duration \& Repetition}]}
		      \item \textbf{Execution Plan:} Provide a rough timeline of deliverables and development plan for each phase, indicating if approach is \emph{repeated-drafting} or \emph{divide-and-conquer}. \textsuperscript{[\hyperlink{projphase}{Phases \& Submissions}]}
		      \item \textbf{Visual Storyboard:} A detailed storyboard (minimum 9 panels), if your project is narrative work, or a mood board (minimum 10 images/videos) if your piece is non-narrative or does not follow conventional linear storytelling. These can be either hand-drawn, digital, mixed-media collages, or audiovisual drafts. If time permits, explore a combination of these methods to communicate your concept as clearly as your mental image. \textsuperscript{[\hyperlink{projideas}{Some Ideas}]}
		      \item \textbf{Audio Samples:} Submit the required number of audio pieces to convey the mood and characteristics of your piece as outlined in your proposal. \textsuperscript{[\hyperlink{projideas}{Some Ideas}]}
		      \item \textbf{Artist List:} Provide a diverse list of artists from different practices and ``scenes'' whose work you find inspiring in realizing your own project. These can be curated based on aesthetics, technique, software, and/or concept.
		      \item \textbf{Toolbox:} Submit a list of software, tools, and techniques you plan to use or want to explore and experiment with.
	      \end{itemize}
	      \small{\textbf{Note:} As you progress through your project, we understand that things change. Creative work is ``part accident, part intention''. This proposal enables us to better assist you in realizing your project and to follow and track your progress along the way. It is not a binding contract, so don't worry if things change.}

	\item \itemHead{Exercise}[Re-re-narrativize][Due \dSun] \newline
	      In discussion with your classmates, you are to swap your footage from previous exercise with one of your peers. Repeat the process from the previous exercise with this new footage. We will compare and contrast how your reimagining of the audio for the same footage is different from that of your classmates. Upload you videos to Youtube or Vimeo and submit the link to \discordE

	\item \itemHead{Sample Library}[Additions][Due \dSun] \newline
	      Submit a minimum of 2 additional field recordings to the \href{\samplelibPermURL}{sample library}.
	      \begin{itemize}
		      \item \textbf{Filename:} \texttt{[TYPE] Recording CLEAN (Firstname Lastname)}
		      \item \textbf{Specs:} 24bit 44100hz WAV File
		      \item \textbf{Track Submission:} Submit your file names to \discordS
	      \end{itemize}
\end{itemize}