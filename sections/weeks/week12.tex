\def\dMon{3/29}%
\def\dTues{3/30}%
\def\dWed{3/31}%
\def\dThur{4/1}%
\placeDate

\def\tues{11/10}%
\def\thur{11/12}%
% 
\begin{itemize}[noitemsep,topsep=0pt,leftmargin=*]
    \item \textbf{Lessons:} Physics: Cloth
    \item \textbf{\textsc{Exercise 7} - Project 3 Proposal (Due \tues)} \hypertarget{p3prop}    Submit a proposal in PDF format for \hyperlink{project3}{Project 3}. As usual, this proposal is non-binding and subject to change. Nevertheless, it allows us to understand your concepts and ideas more tangibly. Your proposal should contain a minimum of:
    \begin{itemize}
        \item \textbf{A 9-panel storyboard}: Your storyboard can be either hand-drawn, digital, or prototyped using 3D software. If time permits, explore a combination of these methods to push your sketches closer to your mental image. Your storyboard should plan for \textbf{a minimum of 3 minutes of animation}.
        \item \textbf{500-word written statement}: In your written response, explain your concept, idea, and mood of the piece, how do you plan to execute it, and what software(s) and techniques are you planning to use. Remember to plan and account for your project based on the \hyperlink{project3}{requirements}. For example ``\emph{I'm creating a piece about our shifting consuming habits from stores and physical entities to online websites and reseller\dots To achieve this I will create a particle simulation of the most purchased items on Amazon during October, and simulate this in Blender. I will use After Effects to motion track a video of Westland Mall and to composite my animation on top of this video.}''
        \item \textbf{Sounds}: Describe your use of sound in the written statement.
        \item \textbf{References and mood board}: You should include a mood board (minimum 10 images/videos) in your proposal to convey your vision more clearly. Use these references to communicate your piece's mood, concept, format, and style among others.
    \end{itemize}
    \item \textbf{Screenings/Artists:} Sara Ludy, Ian Cheng, Bunny Rogers
\end{itemize}