\hypertarget{longproject}{\section{Project}}


\paragraph{Note:} In light of the pandemic and its imposed limitations regarding space, fabrication, and occupying space, the project has been designed to permit engagement with the concepts and tools of this course without compromising your physical and mental health, your peers', families', and faculties'. To this end, those interested can use their preferred visual software and medium, as well as other familiar avenues to explore and engage with this class from the visual aspect. Do note that all of our activities, readings, assignments, and the course in general, are structured so that your efforts throughout the semester build towards your first and final phases of your project.

\subsection{Description}

Throughout the semester, you will work towards realizing one semester-long project. The work associated with this undertaking is divided into two phases. The first phase of this project is tentatively designated as \ul{\textbf{prototype}} and is due by the semester's 8th week. The second and \ul{\textbf{final}} phase is aimed at \emph{maturing, enhancing, and extending} your pieces while crafting fine details and adding intricacies to your work.

\begin{itemize}
    \tightlist
    \item \itemHead{Project}[Prototype Submission][Due Week 8]
    \item \itemHead{Project}[Final Submission][Due Week 8]
\end{itemize}

\hypertarget{projformat}{\subsection{Medium \& Format}}

The primary limiting factor in creating your semester-long project requires that your piece falls under the \textbf{``moving image''} umbrella term. While this prevents you from engaging with certain mediums, it allows you to imagine your project in any of the following formats:

\begin{itemize}
    \tightlist
    \item Film
    \item Video art
    \item Music videos
    \item Videogames
    \item Interactive media
    \item Generative media
    \item Web-based experiences
    \item Audiovisual performances
    \item System-based and simulation media
    \item Other screen-based/projected/interactive media
\end{itemize}


Work that subverts or transgresses the aforementioned categories should be explained and appropriately communicated in project proposals. Depending on your medium of choice, your projects must be accessible to the rest of the class. This requires that \ul{interactive material such as games and web-based media must be distributed in formats that allow your peers and cohort to experience them as intended}. In addition, interactive media must be accompanied by video documentation during submissions.


\hypertarget{projduration}{\subsection{Duration \& Repetition}}


The length of your projects can vary depending on your concept, execution, and medium; however, there is a \ul{\textbf{hard limit of 2 minutes minimum}}. You are free to extend your projects' length, especially if you plan to work with durational forms of moving images such as film and video art. However, note that you are encouraged to complicate and break the temporal cliches by creating \ul{\textbf{looping, indefinite, and generative}} media.

\hypertarget{projstyle}{\subsection{Sound \& Image}}

It goes  without noting, but the realization and assessment of this project are intrinsically linked to how sound and moving images operate in your project. Use sound to \emph{complicate, complement, and contemplate} your moving images. Your composition of sound and visuals must be \emph{coherent, intentional, and appropriate} so that it can \emph{clearly communicate your concept, ideas, and intentions}. Your sounds and images must \emph{exert control} over each other and your audience. In many ways, they can \emph{accentuate, emphasize, and desensitize} the other half. Each half can \emph{mask, enhance, and direct} the other. Moreover, they can \emph{limit or extend the boundaries} of the imagination and depicted realities as they collaborate to \emph{evoke, provoke, interact and entertain} your audiences. \textbf{\emph{Think of this project as a tango of sound and moving images}}.

\hypertarget{projideas}{\subsection{Some Ideas}}

Think about the reverberation of atoms,\superlink{1}[https://www.youtube.com/watch?v=ZmLvvTzohno] the humming of galaxies,\superlink{1}[https://www.youtube.com/watch?v=NiwUc0Lxak8]\superlink{2}[https://www.nasa.gov/content/explore-from-space-to-sound] a bullet in slow motion,\superlink{1}[https://www.youtube.com/watch?v=QfDoQwIAaXg] or the orchestra played by the organisms contained in an alien body.\superlink{1}[https://www.cbc.ca/news/technology/sound-of-bacteria-swimming-synthesized-by-scientists-1.3252557] Think about events and happenings in isolation or as compounds, e.g., the cracking of a glass jar\superlink{1}[https://www.youtube.com/watch?v=mB6dvL_dmjY] vs. the demolition of a building\superlink{1}[https://www.youtube.com/watch?v=ggg3C87UVCY]. Think about the role sound and images play in conveying time and duration: the hyperlapse of a corpse flower,\superlink{1}[https://www.youtube.com/watch?v=6ly9qL-64EI] which requires 7 to 10 years of growth before blooming for the first time,\superlink{1}[https://en.wikipedia.org/wiki/Amorphophallus_titanum] or the first microseconds of a nuclear explosion.\superlink{1}[https://www.sciencephoto.com/media/928895/view]\superlink{2}[https://interestingengineering.com/filming-the-first-milliseconds-of-a-nuclear-explosion-with-the-rapatronic-a-1950-engineering-marvel]\superlink{3}[https://en.wikipedia.org/wiki/Rope_trick_effect]

What do these phenomena look like? What do they sound like? How can we render them experiential? Now take all of these thoughts and reimagine them with your own voice, touch, and expression in a medium that you are comfortable with and excited to explore. A short documentary, an experimental video, or a looping narrative? Fantastic! Interactive media, videogames, web-based art, or live audiovisual performance? Exciting! Soul-soothing and soul-sucking, tender and violent? Count us in!


\hypertarget{projphase}{\subsection{Phases \& Submissions}}

As creators, you are to decide how you allocate time and submit appropriately, depending on your approach and your project's requirements. You can choose to realize your project in a draft and final phase, whereby you use the first half of the semester to submit a completed-yet-rough version of your project and use the second half of the semester to add detail and bring it to maturity. Alternatively, you can divide the execution process into two. For example, suppose you are working on a short film or documentary. In that case, you can choose to shoot, edit and compose all your footage during the first half of the semester while working on sound, visual effects, and final touches during the second half of the semester. Regardless of the path you take, be it \emph{repeated-drafting} of your project from start to finish, or to \emph{divide-and-conquer} major aspects of it, each phase requires that you \ul{\textbf{submit completed and substantial work}} such that it represents half of the effort you deem necessary in perfecting your concepts, ideas, and pieces.

\subsection{Assessment}

You are to demonstrate a creative and critical engagement with the course material: techniques, theory, and concepts, as demonstrated and discussed in workshops, readings, presentations, and screenings.  Your mastery and articulation, your creative, innovative, and imaginative approach, as well as your personal voice and touch, in addition to the elegance and eloquence of your projects, are deciding factors in the assessment of your pieces.

